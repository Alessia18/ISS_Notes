\chapter{Cryptographic techniques
for cybersecurity}


\section{Cryptography}

\begin{figure}[h]
    \centering
    \includegraphics[page = 2,trim = 0.5cm 3cm 1cm 5cm, clip, width = 0.55\textwidth]{\slides}
\end{figure}


The most used technique to achieve protection for many centuries is \textbf{cryptography}; a mathematical technique that involves algorithms for encryption and decryption:
\begin{itemize}
    \item the encryption algorithm takes a message (in clear) and transforms it in such a way that it becomes unintelligible;
    \item to recover the original text, the decryption algorithms make it readable again.
\end{itemize}
Next to the algorithms, \textbf{key-1} is needed for encryption and \textbf{key-2} for decryption, both of which are streams of bits.
Cryptography is used in communication and for data storage (for example, to store data on disks without permission to read them except for authorized users). The common terminology used in cryptography includes two other keywords:
\begin{itemize}
    \item \textbf{Plaintext} or \textbf{cleartext}: the unencrypted message, typically referred to as \textbf{P};
    \item \textbf{Ciphertext}: the encrypted message, typically referred to as \textbf{C}. Note that in some countries, the term "encrypted" may sound offensive for religious reasons (related to the cult of the dead); in such cases, "\emph{enciphered}" is preferred.
\end{itemize}


\subsection*{Cryptography's strength (Kerchoffs' principle)}
Kerckhoffs' Principle (1883) states that \ul{the security of a cryptosystem must lie in the choice of its keys only};
everything else (including the algorithm itself) should be considered of public knowledge. 
However, this principle relies on the fact that the keys have the following properties.
\begin{itemize}
    \item Are kept \textbf{secret};
    \item Are managed only by \textbf{trusted systems};
    \item Are of \textbf{adequate length} % TODO: cross-reference this
\end{itemize}

If these properties are met, 
not only it has no importance that the encryption and decryption algorithms are kept secret, but it is better to
make the algorithms public so that they can be widely analysed, and their possible flaws and vulnerabilities identified.

\subsection*{Security through obscurity (STO)}
The Kerckhoffs' Principle is related to the concept of \textbf{Security through obscurity}: it means that a system is protected, but the details on how it has been protected are not disclosed.\\
Generally, \ul{this alone is not considered a valid security mechanism} because if someone discovers how the system has been protected (and we have seen that there are also non-technical ways by which this can be achieved), it is no longer secure. 

For this reason, we say that \emph{"Security through obscurity is as bad with computer systems as it is with women"}.
\begin{quote}
    "Men try to hide things from women, but when they discover the truth, it is worse than if they had known it from the beginning"
\end{quote}
\begin{flushright}
    (Antonio Lioy)
  \end{flushright}
\emph{Editor's note:} don't hide things from your partner, regardless of gender.


However, there is a category of people (such as military men) which tend to apply STO, but as an \ul{additional
layer}. It is possible to use STO as a layer only if a really strong algorithm is used (but not a secret one).



\section{Symmetric and asymmetric cryptography}
Depending on which relation exists between key-1 and key-2 there are different kinds of cryptography.

\subsection*{Secret key / symmetric cryptography}
\begin{figure}[h]
    \centering
    \includegraphics[page = 6,trim = 2.5cm 3cm 2.5cm 13cm, clip, width = 0.55\textwidth]{\slides}
\end{figure}

It is so named because \textbf{only a single key} is shared by the sender and receiver. In the diagram, there is a plaintext used as input for the encryption (E) block, along with the key. The result is a comprehensible text that is transmitted to the receiver. To retrieve the original text, the decryption (D) block algorithm is used with the same key that was used for encrypting the initial text. If a different key is used, an output is generated, but it will be incorrect (and typically understandable). 

The issue in the diagram is represented by the dashed line: how can the key be securely shared between the sender and receiver? The formulas used are as follows.

% TODO: fix these formulas 
\begin{align*}
K_1 &= K_2 = K\\
C &= enc(K,P)   & or  C = \{P\} K \\
P &= dec(K,C) = enc^{-1} (K,C)
\end{align*}